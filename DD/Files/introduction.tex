\subsection{Purpose}
The purpose of this document is to provide more technical and detailed information about the software discussed in the RASD document. The Design Document is a guide for the programmer that will develop the application in all its functions. The document will explain and motivate all the architectural choices by providing a description of the components and their interaction. We will also enforce the quality of the product through a set of design characteristics. Finally we describe the implementation, integration and test planning.\\
The topics touched by this document are:
\begin{itemize}
	\item high level architecture
	\item main components, their interfaces and deployment
	\item runtime behavior
	\item design patterns
	\item more details on user interface
	\item mapping of the requirements on the components of the architecture
	\item implementation, integration and test planning
\end{itemize}

\subsection{Scope}

CLup is a system that allows customers to line up in a virtual first in first out queue, in order to avoid overcrowding outside of stores. Customers can queue up remotely or on premise (by using a device installed outside the store). When a customer queues up remotely he/she can choose to line up immediately or to book a future visit. The system alerts customers when it is time for them to depart to reach the store. The system builds statistics on customer entry and exit in order to provide a better estimation of waiting times. The system allows the store owners to control the occupation of each of their stores. This is just a summary of all the features of the system, for a more detailed description of the software functionalities read the RASD. 
\newpage
\subsection{Definitions, Acronyms, Abbreviations}
\subsubsection{Definitions}
\begin{tabular}{ | m{5cm} | m{10cm} | }
	\hline
	Reservation & Virtual or physical artifact used to identify the position of a customer in a queue \\
	\hline
	Queue up & Customers are lined up in a FIFO queue\\
	\hline
	Enqueued & A customer is enqueued when he has provided the system with a means of identification and requested a reservation\\
	\hline
	Authorized & A customer is authorized when he has been enqueued and is allowed temporary access to the store.\\
	\hline
	Occupation & Number of customers currently present in the store\\
	\hline
	Printer & Device that can read a social security card and print tickets that contains a progressive number and an estimate of the waiting time.\\
	\hline
	User & Either a customer or a store owner.\\
	\hline
\end{tabular}
\subsubsection{Acronyms}
\begin{tabular}{ | m{5cm} | m{10cm} | }
	\hline
	RASD & Requirement Analysis and Specification Document \\
	\hline
	GPS & Global Positioning System \\
	\hline
	S2B & Software to be \\
	\hline
	UI & User Interface\\
	\hline
	FIFO & First in first out\\
	\hline
\end{tabular}
\subsubsection{Abbreviations}
\begin{tabular}{ | m{5cm} | m{10cm} | }
	\hline
	Gn & Goal number n \\
	\hline
	Rn & Requirement number n \\
	\hline
\end{tabular}
\subsection{Revision history}
Ver.1.0 : 7 Jan 2021
%TODO Change also in RASD
\subsection{Reference Documents}
\begin{enumerate}
	\item IEEE Std 830-1998 Recommended Practice for Software Requirements Specifications
	\item Specification Document: R\&DD Assignment A.Y. 2020/2021
	\item uml-diagrams.org
\end{enumerate}
\subsection{Document Structure}
\begin{itemize}
	\item Chapter 1: gives an introduction about the Design Document, enumerating all the topics that will be covered. Moreover this section contains specifications such as the definitions, acronyms, abbreviation,	revision history of the document and the references.
	\item Chapter 2: contains the architectural design choices, with an in-depth look at the high level components and their interactions. It include several views: the component view, the deployment view and the runtime view. Here are described the interfaces (both hardware and software) used for the development of the application, their functions and the processes in which they are utilized. Finally, there is the explanation of the architectural patterns chosen with the other design decisions.
	\item Chapter 3: this section provide an overview of how the user interfaces of the system will look like.
	\item Chapter 4: This chapter maps the requirements defined in the RASD to the design elements defined in this document. 
	\item Chapter 5: here we define the plan for the implementation of the subcomponents of the system and the order in which the integration operations and their testing will be performed.
	\item Chapter 6: shows the effort which each member of the group spent working on the project.
	\item Chapter 7: includes the reference documents.
\end{itemize}

