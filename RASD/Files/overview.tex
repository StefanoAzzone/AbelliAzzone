\subsection{Product perspective}
The S2B will be used by two kinds of users: customers and store owners. The system will provide customers with a simple UI, in order to be accessible for all demographics. To use the software the customers need either a smartphone or a computer (a printer may be required in this case to print a QR code on paper) with a working internet connection. The software will use a location API (e.g. Google Maps) to pinpoint the location of the customers. To allow on premise reservations the store must install a monitor and a ticket printer outside of the store, and connect them to the system.\\
The system offers to each of the users different functionalities:
\subsubsection{Customer}
The customer can create reservations for a desired store: such reservation is then inserted into a queue. This queue is managed in order to avoid overcrowding, and to serve customers with a first come first served policy (FIFO queue). To achieve this, statistics about visits' durations are exploited, taking also into account information that customers may provide (e.g. which department of the store they want to visit).\\
There are multiple types of reservations that can be requested:
\begin{enumerate}
	\item Immediate reservation: the costumer wants to queue up immediately to a store. He/she is also provided with an estimation of the time he has to wait before his/her turn.
	\item Future reservation: the costumer wants to book a future visit at a desired time and store.\\ When creating the reservation the customer can specify how long he/she intends to stay and which departments of the market he/she plans to go to, in order to provide a better plan. If he/she does not specify the duration the system can infer it using some statistic built on his/her previous visits. Then the customer will be provided whit the actual time he will be able to access the store (considering the bookings from other customers).
	\item On premise reservation: the customer is allowed to enqueue directly at the store. Each of them will provide a system to print tickets so that those who do not have access to the required technology for the previous options can still line up. These tickets contain a reservation number and an estimation of the waiting time before being able to enter.
\end{enumerate}
Each customer will be assigned a number that represents his/her position in queue. A monitor outside the store will display the number of the last customer allowed to enter.
Customers who requested an immediate or a future reservation can receive an alert when they need to depart to reach the store through their smartphone app.
This alert is based on the location of the customer, so that the time needed to reach the store is taken into account. If, for any reason, a customer cannot to go to a store he/she has lined up to it is possible for him/her to delete the immediate or future reservation. The on premise reservation cannot be deleted.\\\\
Reservations can be in one of the following states:
\begin{itemize}
	\item Pending: customer cannot yet access the store
	\item Authorized: customer can access the store
	\item Current: customer is in the store
	\item Expired: customer has exited the store or the reservation has been canceled
\end{itemize} 
\subsubsection{Store owner}
The store owner can
\begin{enumerate}
	\item Register a store to the system so that it can be accessed by customers
	\item Set the maximum occupation threshold for each of his stores.
	\item Monitor the number of customers inside at any time
	\item Visualize statistics about the flow of clients and department occupation
\end{enumerate}
When setting the maximum occupation of a store, the owner can specify a threshold for each department or for the whole store. In the latter case the occupation of each department is set to the same value. In case customers select which sections of the store they intend to visit their presence will be considered only in those specific departments, in order to allow more customers in, provided that they will not come in contact with each other. For safety, if customers don't specify anything they are supposed to visit each department.
\begin{figure}[!htb]
\centering
\includegraphics[width=\textwidth]{Images/ClassDiagram.png}
\caption{\label{fig:metamodel2}Class Diagram.}
\end{figure}
\newpage
\subsubsection{State charts}
Here we show the main processes that the system will manage, and the states in which the system will find itself.
\begin{figure}[!htb]
	\centering
	\includegraphics[width=\textwidth]{Images/StateDiagram1.png}
	\caption{\label{fig:metamodel3}State Diagram 1: Creation of a reservation.}
\end{figure}\\
%TODO add by car/on foot request
The figure above illustrates how the customer creates a reservation (virtual customer). The customer accesses the main menu. From there he/she can request an immediate reservation or a future reservation. If he/she requests an immediate reservation, it is created and the process terminates. If he/she requires a future reservation he can add the departments he/she intends to visit, and how much time he/she is going to spend at the store. Then the reservation is created and the process terminates.\\\\\\\\

\begin{figure}[!htb]
	\centering
	\includegraphics[width=\textwidth]{Images/StateDiagram2.png}
	\caption{State Diagram 2: Store owner sets occupation.}
\end{figure}
The figure above illustrates how the store owner sets the maximum occupation of one of his stores. The store owner logs in and is provided a main menu. He/she selects one of his stores, and is presented with the current maximum occupation of his selected store. He can change it to a value to his liking, or go back to the main menu.
\newpage

\begin{figure}[!htb]
	\centering
	\includegraphics[width=\textwidth]{Images/StateDiagram3.png}
	\caption{State Diagram 3: On premise creation of a reservation.}
\end{figure}
The figure above illustrates how a customer can queue up on premise (physical customer). The customer is presented with an on premise main menu. From there he/she can request to queue up (create a reservation) immediately. He will be prompted to identify himself (e.g. social security card). Once identified the customer will be provided a ticket with a number that represents his position in queue and an estimate of the waiting time.

\begin{figure}[!htb]
	\centering
	\includegraphics[width=\textwidth]{Images/StateDiagram4.png}
	\caption{State Diagram 3: Customer enters a store.}
\end{figure}
The figure above illustrates how a customer who is authorized can access the store. At the entrance of his/her store of choice he is enabled to demonstrate his authorization. Once he/she does, he is allowed to enter the store.

\subsubsection{Scenarios}
\paragraph{Customer queues up remotely}
Gordon comes home from work at 5:30 PM. He would like to go to the store. He is a registered user of CLup. His smartphone has internet connection and GPS on. He opens the CLup mobile application and requests an immediate reservation to a store of his choice, selecting the car as means of transport. After some time he is alerted that he needs to depart to avoid being late at the store and losing the reservation. Gordon departs and arrives to the store. He opens the application and selects the reservation he created which now contains the number that represents his position in the queue. After some time the monitor outside the store shows his number, which means it is his turn to enter. He activates the turnstiles with the QR now shown in the reservation page of the mobile application and enters the store. He buys all the products he needs. He opens the application and selects the reservation he created which now contains a QR code. He activates the turnstiles and exits the store.
\paragraph{Customer queues up on premise}
Marco is an old man. He does not have a smartphone. He would like to go to the store next to his house on foot. He reaches the store, and retrieves a queue reservation ticket from the printer outside the store, by inserting his social security card. The ticket displays a QR code, a number that represents his position in queue and an estimated waiting time. He can now come back home and wait the specified amount of time, then come back to the store. At that point the monitor outside the store shows his number, which means it is his turn to enter. He activates the turnstiles with the QR printed on his ticket and enters the store. He buys all the products he needs. He activates the turnstiles with his ticket and exits the store.
\paragraph{Store owner registering a store and setting occupation}
Paolo is the proud owner of the famous supermarket chain "Junes" and today there is the first opening of a new store of the chain. Due to Covid restrictions, he already use CLup to avoid overcrowding inside his stores. He decide to use the service also for the new building, so he access the system through his PC. Once logged in he register his new supermarket by inserting all necessary data. Then he select the newly registered store and set an appropriate occupation for each of its department, based on how many people it can contain, ensuring social distances.
\paragraph{Customer books a visit}
Lara is very organized woman; she likes to manage situations in advance, and for this reason, in the recent period, she uses CLup to book her visit to the store. Every Sunday she open the CLup web page from her phone, logs in and create a new reservation for the next Saturday. Because she request this reservations in advance af almost a week, she usually has no problems to book exactly for the time she prefer, so she is able to get an appointment for the 9.30 and print the ticket. In this way when the time arrives she can enter the store with no delays and rows.

\subsection{Product functions}
We describe here functions that the system will support.
\begin{enumerate}
	\item Customers can queue up virtually or physically, immediately or in the future at one of the stores registered to the service.
	\item Customers can delete a virtual reservation.
	\item Customers can access the desired store when its occupation reaches acceptable levels.
	\item Customers that queued up virtually can be notified when it is time for them to depart to reach the store.
	\item Store owners can monitor the occupation in each of their stores.
	\item Store owners can define the maximum occupation of each department of their stores.
	\item Store owners can register stores to the system.
\end{enumerate}

\subsection{User characteristics}
\begin{enumerate}
	\item {\bfseries Customers}: a physical person that needs to access any of the stores registered to the system. The customers belong to all demographics, thus the need for a user-friendly interface for both virtual and physical customers. The customer needs a reasonably precise estimate of waiting time and time to get to the store.
	\item {\bfseries Store owners}: a physical or legal person that owns any number of stores and needs to enable customer access through a queue system.
\end{enumerate}

\subsection{Assumptions, dependencies and constraints}
\subsubsection{Domain Assumptions}
\begin{enumerate}[label=D\arabic*]
	\item The stores have QR activated turnstiles.
	\item Turnstiles let one and only one person in each time they unlock.
	\item Outside stores is a social security card activated ticket printer.
	\item Outside stores there is a monitor.
	\item There is no way for a customer to enter a store except from entrance and exit.
	\item Each customer has either a telephone number or an identification document.
	\item When provided, user location has maximum error of 5 meters.
	\item To register to the S2B users must have either a smartphone or a computer.
	\item To register and use the S2B users must have an internet connection.
\end{enumerate}